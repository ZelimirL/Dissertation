\chapter{Radiative Charge Transfer}

TODO: Reference the Mathematica/Fortran code also.

\section{Non-Adiabatic Charge Transfer}

Similarly to the Radiative Association, this is another scattering process.  In this case, we do not have the formation of a molecule as a results. Instead, we have an electron transition from the  $ A^2\Sigma^{+} $ to $ X^2\Sigma^{+} $ level, accompanying by the emission of the photon. Since no stable state is formed, we have two partial waves, one incoming and one leaving the scattering area.

\subsection{Description}

The cross section of the spontaneous radiative association is again given by \cite{Zygelman1}

\begin{equation}
\begin{split}
& \sigma_{CT}  = \int_0^{\omega_{max}}{d\omega\,\frac{d\sigma}{d\omega}} \\[.8em]
& \sigma_{sp}(k) = \sum_J\sum_n{\frac{64}{3}\frac{\pi^5\nu^3}{c^3k^2}\left[(J+1)M_{J+1,J}^2 + J\,M_{J-1,J}^2 \right] }
\end{split}
\end{equation}

where
\begin{equation}\label{dint1}
\begin{split}
& \frac{d\sigma}{d\omega} = \sum_{J}{\frac{8}{3}\frac{\pi^2\omega^3}{c^3k^2} \left[J\,M_{J,J-1}^2(k,k') + (J+1)M_{J,J+1}^2(k,k')  \right]}
\end{split}
\end{equation} 
and
\begin{equation}\label{dint2}
\begin{split}
& M_{J,J'}(k,k') = 
\end{split}
\end{equation} 

\subsection{Calculation}



