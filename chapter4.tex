\chapter{The Radiative Association in Ion Atom Collisions in 2D Space} 

Radiative Association is a reactive process in which two particles (atoms, or ionized molecules) collide to form a large(r) molecule while emitting a photon. The simplest such case is the formation of the hydrogen molecular ion, in the process $ H^{+} + H \rightarrow H_2^{+} + \hbar\omega $.

For the radiative association process to be efficient, it is necessary that the molecules form a stable state, and that it is able to shed the excess energy by emitting a photon \cite{Zygelman89},\cite{Zygelman1} . Therefore this can be modeled as a scattering problem, and moreover a scattering at a low energy and non-relativistic. I am not aware of any experiments regarding charge transfer in 2D, so at this point the results below would benefit from the experimental confirmation.  

In 3D, the radiative association of (cold) atoms and ions ccurs in astrophysics, TOOD: References  and they are considered to be responsible for the molecular synthesis in the interstellar clouds. In addition to strictly Hydrogen reactions, the process that leads to the formation of $ HeH $ molecule have been well analyzed in the past \cite{Zygelman1}. This has lead to the analysis of the polyatomic, complex systems, most recently the interaction of ultra cold $ Yb^{+} $ ion with the $ Ca $ atoms. However, in the case of $ H_2 $ molecule, the 2 dimensional case has not been considered yet. 

\subsection{Theory} 

In this thesis we will calculate the cross section and the emission spectra of the reaction $ H^{+} + H \rightarrow H_2^{+} + \hbar\omega $.

The restriction is that atoms are confined in 2 dimensions, while the radiation is not. Therefore the photon can be emmited in any direction and the potential felt by the incoming ion is the standard Coulomb potential.

Now, this is a scattering problem, with a twist of a Born-Oppenheimer approximation. The Hamiltonian in this case will contain another term, namely the interaction of the radiation field with the electron.  Following \cite{Zygelman88}\cite{Zygelman89},  in the center of the mass of the nuclei, the Hamiltonian for the system is given by: 

\begin{equation}\label{eqH} 
\begin{split} 
& H = -\frac{1}{2\mu}\nabla^2_{\mathbf{R}} + H_{el}(\mathbf{R},\mathbf{r}) + H_{rad} + H_{int} 
\end{split} 
\end{equation} 

where $ \mu $ is the reduced mass, $ nabla_{\mathbf{R}} $ is the gradient operator for the relative nuclear motion. $ H_{el}(\mathbf{R},\mathbf{r}) $ is the fixed nuclei Hamiltonian for the electron, whose coordinate are labeled by $ \mathbf{r} $. $ H_{rad} $ is the Hamiltonian of the radiation field, and $ H_{int} $ is the radiation-matter coupling. Since we are dealing with the interaction of an atom with the EM radiation, it is common to use the length gauge.  

The length gauge is a gauge transformation that replaces the vector potential for the field by the scalar potential for the quasi-static electric field \cite{LengthGauge3}.  In this gauge we take the Hamiltonian as: $ H = \mathbf{p}^2/2m + V(\mathbf{r})  + e\mathbf{E}\mathbf{r} $. The length gauge is convenient since both the Coulomb and the external fields are represented by the scalar potentials, which are additive. In the presence of the radiation field, the length gauge is obtained by the gauge transformation of the vector potential $ \mathbf{A} $, such that $ \mathbf{A} \rightarrow \mathbf{A} + \nabla \chi $ where $ \chi = - \mathbf{r} \cdot \mathbf{A} $. 

In the length gauge, the interaction Hamiltonian is:
\begin{equation}
\begin{split}
& H_{int} = -\sum_j{ \mathbf{r}\cdot\mathbf{E} } \\[.8em]
& \mathbf{E} = i\,\sum_{k\alpha}{\left(\frac{2\pi c k}{V}\right)^{1/2}\hat{\epsilon}_{k\alpha}\left(a_{k\alpha} - a^{\dagger}_{k\alpha}\right)}
\end{split}
\end{equation}
where $ a_{k\alpha} $ and $ a^{\dagger}_{k\alpha} $ are destruction and creation operators for the photon of momentum $ \hbar k $ and polarization $ \alpha $ respectively.

TODO: Verify charge transfer vs resonant charge transfer
So if we regard this process as a transition induced by the radiation field from the $ A^{1}\Sigma^{+}_u $ state to of the $ H_2^{+} $ molecule formed by the approaching atom and an $ H $ ion, to the $ X^{1}\Sigma^{+}_g $ state in which the atom and the ion separate.

Now we write the system wave function:

\begin{equation}\label{ansatz}
\begin{split}
& \Ket{\Psi} = F_a(\mathbf{R})\chi_a(\mathbf{R},\mathbf{r})\Ket{0} + \sum_{k\alpha}{F_{k\alpha}(\mathbf{R})\chi_b(\mathbf{R},\mathbf{r})\Ket{k\alpha}}
\end{split}
\end{equation}

where $ \chi_a(\mathbf{R},\mathbf{r}) $ and $ \chi_b(\mathbf{R},\mathbf{r}) $ are the eigenstates of the fixed position nuclei Hamiltonian $ H_{el} $ corresponding to the $ A^{1}\Sigma^{+}_u $  and $ X^{1}\Sigma^{+}_g $ state in body fixed frame respectively. The $ F_a(\mathbf{R}) $ and $ F_{k\alpha}(\mathbf{R}) $ are the amplitudes for the relative nuclear motion and $ \Ket{0} $ and $ \Ket{k\alpha} $ are the kets for the photon vacuum and single photon states. The ansatz \eqref{ansatz} is valid at low speed collisions, where the other channels are unaccessible. In the adiabatic approximation (i.e. ignoring the non-adiabatic effects) the amplitudes $ F_a(\mathbf{R}) $ and $ F_{k\alpha}(\mathbf{R}) $ obey the set of coupled equations:

\begin{equation}\label{Fk}
\begin{split}
& \left[-\frac{1}{2\mu}\nabla^2_{R} + V_a(R) - E\right]F_a(\mathbf{R}) = \sum_{k\alpha}{F_{k\alpha}(\mathbf{R})U_{k\alpha}(\mathbf{R}) } \\[.8em]
\end{split}
\end{equation}
\begin{equation}\label{Fka}
\begin{split}
& \left[-\frac{1}{2\mu}\nabla^2_{R} + V_b(R) + \hbar\omega - E\right]F_{k\alpha}(\mathbf{R}) = F_a(\mathbf{R})U^{\dagger}_{k\alpha}(\mathbf{R}) 
\end{split}
\end{equation}
where

\begin{equation}
\begin{split}
U_{k\alpha}(\mathbf{R}) = -i\left[\frac{2\pi c k}{V}\right]^{1/2}D(R)\hat{\mathbf{R}}\cdot\hat{\mathbf{\epsilon}}_{k\alpha}
\end{split}
\end{equation}
and $ V_a(R), V_b(R) $ are the potential energy curves for the $ A^{1}\Sigma^{+}_u $ and $ X^{1}\Sigma^{+}_g $ states respectively.   $ D(R) $ is the radial transitional dipole moment between them. $ E $ is the initial energy of the relative motions and $ \omega $ is the angular frequency of the emitted photon. 

Now to solve this. We find the Green function for the equation \eqref{Fka} which satisfies the retarted boundary conditions so that $ F_{k\alpha} $ contains only outgoing waves in the limit $ R \rightarrow \infty $:

\begin{equation}\label{GreenFkaEq}
\begin{split}
& \left[-\frac{1}{2\mu}\nabla^2_{R} + V_b(R) + \hbar\omega - E\right]G^{+}(\mathbf{R}, \mathbf{R}') =  \delta^3(\mathbf{R}, \mathbf{R}')
\end{split}
\end{equation}
and from the equation \eqref{GreenFkaEq} we get:
\begin{equation}\label{FkaInt}
\begin{split}
& F_{k\alpha}(\mathbf{R}) = \int{d^3R'G^{+}(\mathbf{R}, \mathbf{R}')F_{a}(\mathbf{R}')U^{\dagger}_{k\alpha}(\mathbf{R}')   }
\end{split}
\end{equation}

We can express this function in the partial waves bases. Since $ V_b $ contains no bound states we get for $ \mathbf{R} = \mathbf{R}(R,\theta) $:
\begin{equation}\label{GreenFka}
\begin{split}
G^{+}(\mathbf{R}, \mathbf{R}') =  \frac{\pi\mu}{k_b}\sum_{l=0}^{\infty}{\sqrt{\frac{1}{2\pi}}\cos(l\,\theta)\cos(l\,\theta^{'})}\times \frac{f_l(k_bR_{<})g^{+}_l(k_bR_{>})}{RR'}
\end{split}
\end{equation}
where $ P_l(\cos\theta) $ are Legendre polynomials, also $ m = 0 $ part of the spherical harmonics:  $ P_l(\cos\theta) = Y_{l0}{\theta,\phi} $. 

The $ f_l(k R) $ is a regular solution of the homogenous Schrodinger radial  equation for the 2D case: \cite{H2atom}:
\begin{equation}\label{eqRadial1}
\begin{split}
& \left\{\frac{d^2}{dR^2} - \frac{l(l+1)}{R^2} - 2\mu\left[V_b(R) - V_b(\infty)\right] + k^2\right\}f_l(kR) = 0 \\[.8em]
& k \equiv \sqrt{2\mu[E - \hbar\omega - V_b(\infty)}
\end{split}
\end{equation}
where:

\begin{equation}
\begin{split}
f_{l} \sim \sqrt{\frac{2}{\pi}}\sin\left[kR - \frac{l\pi}{2} + \delta_l(b)\right]
\end{split}
\end{equation}

and $ g^{+}_l(kR) $ is irregular solution with the boundary condition at large $ R $.
\begin{equation}
\begin{split}
g^{+}_{l} \sim \sqrt{\frac{2}{\pi}}\exp i\left[kR - \frac{l\pi}{2} + \delta_l(b)\right]
\end{split}
\end{equation}
and $  \delta_l(b) $ is a phase shift. 

The total wave function \eqref{ansatz} must be symmetric under the interchange of the $ H $ nuclei, so that $ F_a(\mathbf{R}) = -F_a(-\mathbf{R}) $ and $ F_{k\alpha}(\mathbf{R}) =  F_{k\alpha}(-\mathbf{R}) $. Now we solve equations \eqref{Fk} and \eqref{Fka} in the distorted wave approximation. Then $ F_a(\mathbf{R}) $ is the solution of the \eqref{Fk} with the coupling term set to be zero and it can be expressed in the form:


\begin{equation}\label{Flong}
F_a(\mathbf{R}) = \sum_{l=1}^{\infty}{\cos(l\,\theta)(2l+1)i^{l}\sqrt{\pi}\times \exp[i\delta_l(a)]\frac{s_l(k_a\mathbf{R})}{k_a\mathbf{R} } } 
\end{equation}

The asymptotic form for \eqref{Flong} is:
\begin{equation}\label{FlongA}
F_a(\mathbf{R}) \sim \frac{1}{\sqrt{2}}\left[e^{ik_az}-e^{-ik_az} + [f(\theta,\phi) - f(\theta-\pi,\phi+\pi)]\frac{e^{ik_aR}}{R}\right]
\end{equation}

By inserting \eqref{Flong} and \eqref{GreenFka} into \eqref{FkaInt} we get the asymptotic form for the  \eqref{FkaInt}:
\begin{equation}\label{FlongAA}
F_{k\alpha}(\mathbf{R}) \sim \frac{e^{ik_aR}}{R}f_{k\alpha}(\theta)
\end{equation}
where:
\begin{equation}\label{fkaa}
f_{k\alpha}(\theta) = sum_{l}{\cos(l\,\theta)\left[\sum_{J=1}^{\infty}{\frac{2\pi\mu}{k_ak_b}(2J+1)\left(\frac{\pi k c}{A}\right)^{1/2}e^{i\delta_j(b)}e^{i\delta_j(a)}i^{J+l-1} }\right]\sqrt{2l+1)2\pi}}
\end{equation}

In this summation, the $ j $ is restricted to the odd integers, and 

\begin{equation}\label{Mll}
M_{l,l'}(k_a,k_b) = \frac{1}{\sqrt{k_a,k_b}}\int_0^{\infty}{dRs_l(k_a,R)D(R)f_{l')(k_b,R)} }
\end{equation}

The cross section of the collision induced transition between the $ H $ atom and the $ H^{+} $ ion is obtained by summing $ \left|f_{k\alpha}(\theta)\right|^2 $ over all final states that conserve energy with an inital state, and dividing the result with the flux of the incident channel.

The $ H\left(2 \leftidx{^1}{S}\right) \rightarrow H\left(1 \leftidx{^1}{1S}\right) $ is a linear combination of $ A \leftidx{1}{\sigma_u^{+}} $ and $ X \leftidx{1}{\sigma_g^{+}} $ states. Since the excited gerade state is not allowed to make a radiative transition to a gerade ground state, the flux in the incident channel is twice the flux in the $ A \leftidx{1}{\sigma_u^{+}} $ channel.

So for the cross section we get:

\begin{equation}\label{crs}
\begin{split}
& \sigma = \int_0^{\omega_{max}}{d\omega\frac{d\sigma}{d\omega}} = \\[.8em]
& = \sum_{\alpha}{\int{\frac{d^2k}{(2\pi)^2}\frac{A}{2\mu k_a}\int{d^2k_b\delta\left[\frac{k_b^2}{2\mu} - \frac{k_a^2}{2\mu} + \Delta E - \hbar\omega \right]\left|f_{k\alpha}(\theta) \right| } } }
\end{split}
\end{equation}

where
\begin{equation}\label{diffcrs}
\frac{d\sigma}{d\omega} = \frac{8}{3}\left[\frac{\pi \mu}{k_a}\right]^2 \frac{1}{c^3}\omega^3 \sum_{J}{\left[J M_{J,J-1}^2(k_a,k_b) + (J+1)M_{J,J+1}(k_a,k_b)\right] }
\end{equation}

$ \Delta E $ is the energy of the transition at $ R = \infty $ and $ \omega_{max} $ is the maximum frequency of the emitted photon. Expression \eqref{crs} is an equivalent expression to the Fermi's Golden Rule. Equation \eqref{diffcrs} provides the spectrum of the emitted radiation, in addition to the scattering cross section. 

\subsubsection{Optical Potential Method}

The approximation that does not require the integration over the total spectrum is the optical potential method. TODO: Insert optical potential method.

To derive it here we insert the equation \eqref{FkaInt} for the amplitude $ F_{k\alpha}(\mathbf{R}) $ into the equation \eqref{Fk} to obtain the equation for the amplitude $ F_a(\mathbf{R}) $

\begin{equation}\label{opt1}
\begin{split}
\left[-\frac{1}{2\mu}\nabla_R^2 + V_a(R) - E \right]F_a(\mathbf{R}) = \sum_{k\alpha}{d^2R'G^{+}(\mathbf{R},\mathbf{R}^{'})U_{k\alpha}^{\dagger}(\mathbf{R}^{'})U_{k\alpha}(\mathbf{R})F_a(\mathbf{R}^{'})}
\end{split}
\end{equation}

The right hand of \eqref{opt1} contains a complex, non-local potential

\begin{equation}\label{optV}
V(\mathbf{R},\mathbf{R}^{'}) = \sum_{k\alpha}{G^{+}(\mathbf{R},\mathbf{R}^{'})U_{k\alpha}^{\dagger}(\mathbf{R}^{'})U_{k\alpha}(\mathbf{R}) }
\end{equation}

that arises because of the interaction of the electron with the vacuum. Now the real part of this potential induces the shift in the eigenvalue $ V_a(R) $. Since the coupling on an electron with the radiation is weak, we can ignore it and consider only the imaginary part. 

The imaginary part of $ V(\mathbf{R},\mathbf{R}^{'}) $ is an absortive potential, representing a process where electron in excited state emits a photon and decays to a ground state. This potential is non-local, as we take is for $ R = \infty $. In the optical potential aprroxmation, we take replace potential by the local one, whose range is limited to the scattering area. This potential is essentialy classical.

Because the term $ U_{k\alpha}^{\dagger}(\mathbf{R}^{'})U_{k\alpha}(\mathbf{R}) $ appearing in equation \eqref{optV} is real, the optical potential is proportional to the imaginary part of the retarded Green's function, which is expressed as:

\begin{equation}\label{OptGreen}
\text{Im} G^{+}(\mathbf{R},\mathbf{R}) = \pi\sum_{l=0}^{\infty}{\cos(l\, \theta)\cos(l\,\theta^{'})\int_0^{\infty}{dk\,\delta\left[\frac{k^2}{2\mu} - \frac{k_a}{2\mu} + \hbar\omega - Delta E \right]\frac{f_l(kR)f_l(kR^{'})}{R\,R^{'}} }  }
\end{equation}

This result is obtained using the spectral representation of the retarted Green's function and the identity $ 1/(x + i\epsilon) \rightarrow P/x - i\pi\delta(x) $ as $ \epsilon \rightarrow 0 $. Using \eqref{OptGreen} one obtains for the non-local optical potential:

\begin{equation}\label{optV1}
\begin{split}
V(\mathbf{R},\mathbf{R}^{'}) = \frac{i}{2\pi}\sum_{\alpha}\int{d\Omega_k}\int_0^{k_{max}}{\sum_{l=0}^{\infty}{\cos(l\, \theta)\cos(l\,\theta^{'})\times\frac{\omega^3}{c^3}\frac{f_l(kR)f_l(kR')}{R\,R'}D(R)D(R')(\hat{\mathbf{R}}\cdot\epsilon_{k\alpha})(\hat{\mathbf{R}}^{'}\cdot\epsilon_{k\alpha}) } }  
\end{split}
\end{equation}

where $ \omega(k) = k_{\alpha}/2\mu + \Delta E - k^2/mu $.  Now for the optical-potential approximation we make the semi-classical approximation that the values of $ k $ that give the largest contribution are given by:

\begin{equation}
\frac{k^2}{2\mu} \simeq = \Delta E + \frac{k_{\alpha}^2}{2\mu} + V_b(r) - V_a(r)
\end{equation}

Now the frequency term $ \omega^3 = \left|\Delta E(R)\right|^3 $ can now be taken outside the integral.

Using the expansion: TODO: delta function expansion in 2 dimensions and verify below

\begin{equation}
\delta^2\left(\mathbf{R},\mathbf{R}^{'}\right) = \sum_{l=0}^{\infty}{\cos(l\,\theta)\cos(l\,\theta')\frac{\delta(R - R')}{R\,R'}}
\end{equation}

we get:

\begin{equation}
\begin{split}
& V_{opt}(\mathbf{R},\mathbf{R}) \approx = \frac{i}{2}\delta^2\left(\mathbf{R},\mathbf{R}^{'}\right)A(R), \\[.8em]
& A(R) = \frac{4}{3}D^2(R)\frac{\left|\Delta E(R)\right|^3}{c^3}
\end{split}
\end{equation}

and equation \eqref{optV1} becomes:

\begin{equation}\label{optApprox}
\left[-\frac{1}{2\mu}\nabla_R^2 + V_a(R) - E\right]F_{\alpha}(\mathbf{R}) = \frac{i}{2}A(R)F_{\alpha}(\mathbf{R})
\end{equation}

The cross section for the radiative quenching is given by:

\begin{equation}
\sigma = \frac{\pi}{k_a^2}\sum_{J=0}^{\infty}{(2J+1)\left(1-e^{-4\eta_j}\right) }
\end{equation}

where $ eta_j $ is the imaginary component of the phase shift of the $ Jth $ partial wave of the solution \eqref{optV1}. The sum on $ J $ is restricted to the odd integers. 
Also because the right hand of \eqref{optV1} is small, we can use the distorted wave approximation to obtain the expression for the phase shift $ eta_j $

\begin{equation}\label{eta1}
\eta_j = \frac{\pi\mu}{2k_a}\int_0^{\infty}{dR\,\left|s_J\left(k_aR\right)\right|^2A(R) }
\end{equation}

TODO: JWKB

Using the JWKB approximation to the \eqref{eta1}, by replacing the sum with an integral, and recognizing that $ \eta_J $ is small, we obtain the semi-classical cross section:

\begin{equation}\label{sccs}
\sigma = 2\pi\sqrt{\frac{2\pi}{E}}\int{dp\,p}\int_{R_c}^{\infty}{dR \frac{A(R)}{\sqrt{1-\frac{V_a(R)}{E}-frac{p^2}{R^2}}}}
\end{equation}

where $ R_c $ is the classical turning point and $ E $ is the kinetic energy.

As expected, there does not exist the analytic solution to this  differential equation . Ergo, I used the numerical method, namely Mathematica code to solve this equation. This equations behaves  'well' so the solution is easy to find numerically. 

\section{Charge Transfer}

TODO: Make this a new chapter.

TODO: Electronic translation factor (ETF) 

As usual, we also employ the Born-Oppenheimer (BO) approximation. We expand the scattering wave function in the terms of BO wave functions, modifed by the electronic translation factor. 
If we set $ \chi_i^a(\mathbf{R}) $ to the be wave function of the nuclear motion in the electronic state $ i $, we get for the wavefunction:

\begin{equation}
\Psi(\mathbf{r},\mathbf{R}) = \sum_i{exp\left[\frac{1}{\mu}\mathbf{S}\cdot\nabla_R \right]\phi(\mathbf{r},\mathbf{R}) }\chi_i^a(\mathbf{R})
\end{equation}

with $ \mu $ being the reduce mass, and 
\begin{equation}
S = \frac{1}{2}f_i(\mathbf{r},\mathbf{R})\mathbf{r}
\end{equation}

where $ f_i $s are the switching functions that incorporte the molecular character of the ETF.
The equations for the $ \chi_i^a(\mathbf{R}) $ can be obtaind in a matrix form:
\begin{equation}
\left\{-\frac{1}{2\mu}\left[\underline{I}\nabla_R - i(\underline{\mathbf{P}} + \underline{\mathbf{rAP}})\right]^2 + \underline{V}\right\}\underline{\chi}^a(\mathbf{R}) = E\underline{\chi}^a(\mathbf{R})
\end{equation}
with:
\begin{equation}\label{matrixFactors}
\begin{split}
& \mathbf{P}_{ij} = \Braket{\phi_i | -i\nabla_R | \phi_j} \\*
& \mathbf{A}_{ij} = i(E_i - E_j)\Braket{\phi_i | \mathbf{S} | \phi_j} \\*
& V_{ij}(R) = \delta_{ij}V_i(R)
\end{split}
\end{equation}
where $ E $ is the energy of the nuclear motion in the center of mass frame, and $ \underline{I} $ is the identity matrix. The matrix $ \mathbf{P}_{ij} $ represents the non-adiabatic coupling, the $ \mathbf{A}_{ij} $ is the ETF correction, and the $ V_i(R) $ is the potential energy of the $ i $th Born-Opennheimer state.

\subsection{ Radiative charge transfer and radiative association}

The radiative charge transfer cross-section can be calculated using the formula \eqref{crs} for $ \sigma $ with $ M $ calculated using formula \eqref{Mll}, where the $ k_a $ and $ k_b $ are the wave numbers of the initial and final state. 

The partial waves $ f_j(kR) $ and $ s_j(kR) $ are the regular solutions of the homogenou radial equations.

\begin{equation}
\left\{\frac{d^2}{dR^2} - \frac{J(J+1)}{R^2} - 2\mu\left[V_a(R) - V_b(\infty)\right\} + k^2\right\}s_j(kR) = 0
\end{equation}
and
\begin{equation}
\left\{\frac{d^2}{dR^2} - \frac{J(J+1)}{R^2} - 2\mu\left[V_b(R) - V_a(\infty)\right\} + k^2\right\}f_j(kR) = 0
\end{equation}

with $ V_a(R) $ and $ V_b(R) $ being the potential energy curves of the final ground state and the initial excited state, respectively. The $ k $s are the wave numbers, given by:

\begin{equation}
\begin{split}
& k_b = \sqrt{2\mu\left[E - \hbar\omega - V_b(\infty)\right]} \\*
& k_a = \sqrt{2\mu\left[E - V_a(\infty)\right]} \\*
\end{split}
\end{equation}
where $ \omega$ is the angular frequency of the emitted photon, $ E $ is the total collision energy in the center of mass frame.

Again, we get calculate the total cross section given by the radiative decay as:




One has to be careful though, since it seems that most numerical methods are very sensitive to the choice of initial conditions.  As a second order equation, the solution requires two initial conditions, either in the form of function value at the ends of the interval, or value of the function and its derivative at some point. Now, this partial wave represents an incoming particle (electron) and thus the function  $  f(kR) $  is defined on the infinite interval and it is oscillatory for large values of $ kR $. Because of that, it is impossible to specify a boundary condition for value of $  f(kR) $ at infinity. So the boundary conditions have to be in the form of the value of function at some point and the value of the derivative at the same point. 

In my calculation, I assume that  for $ kR \rightarrow 0 $, the function $  f(kR) $ is finite and 'well  behaving', so the appropriate boundary conditions seem to be: 
\begin{equation}
\begin{split}
& \text{ For }\,kR \rightarrow 0,\,f(kR) = 1\,;\,\,f'(kR)  = 0
\end{split}
\end{equation}

The Mathematica code for the $  f(kR) $ is in appendix TODO: add code.

Following the approach in the \cite{ZL} I apply a semi-classical approach to the system of two $ H_2^{+} $ molecules. In this approach, I assume that a photon is emitted with energy equal to the energy difference between two Born-Oppenheimer potential surfaces, at the distance where the transition occurs. This leads to the Local Optical Potential Method, and following the semi-classical approach, the total rate is estimated as a classical integral over all localized transitions. 

The cross section of the spontaneous radiative association is given by \cite{Zygelman1} .  This can also be derived 
using the Fermi's Golden Rule (which turns out to be published by Dirac's 20 years before Fermi):

\begin{equation}
\begin{split}
& \sigma_{CT}  = \int_0^{\omega_{max}}{d\omega\,\frac{d\sigma}{d\omega}} \\[.8em]
& \text{ where }: \\[.8em]
& \sigma_{sp}(k) = \sum_J\sum_n{\frac{64}{3}\frac{\pi^5\nu^3}{c^3k^2}\left[(J+1)M_{J+1,J}^2 + J\,M_{J-1,J}^2 \right] }
\end{split}
\end{equation}

The sum extends over the rho-vibrational quantum numbers $ n $ and angular momentum $ J $ of the $ H_2^{+} $ ion. Due to the topological constraints the direction of $ J $ remains fixed, and only its magnitude changes. As expected, for the ground state we have $ J = 0 $. 

The $ M_{J,J'}  $ is an overlap integral defined by:
\begin{equation}
\begin{split}
& M_{J,J'} = \int_0^{\infty}{dR\,f_j(kR)D(R)\phi_{J'}^n(R)}
\end{split}
\end{equation}

where $ f_j(kR) $ is a partial wave defined above. 

\subsubsection{Vibrational Levels}

The $ \phi_{J',n}(R) $ represent the vibrational eigenfunction of the ground state   $ X^2\Sigma^{+} $ with the energy $ \epsilon_nJ $. TODO: Show picture.   Given the nature of the problem 2D, the ground state has the $ J = 0 $. So the differential equation for the  $ \phi_{0,n}(R) $ , in the potential well, is:
\begin{equation}
\begin{split}
& \phi_{0,n}^{''}(R) + \left[ E(R) + V(R)  \right]\phi(R)_{0,n} = \epsilon_n \phi(R)_{0,n}
\end{split}
\end{equation}

Of course the solution to this equation can only be calculated numerically (appendix TODO: shows the Mathematica code. ).  To solve the equation one must set the boundary conditions to  $ \phi(R)_{0,n} \Rightarrow 0, \phi(R)^{'}_{0,n} \ $ for $ R \Rightarrow 0 $. For actual numerical evaluation, the important boundary condition is the value of  $ \phi(R \approx 0 )_{0,n} $.  With these boundary conditions set, the values of $ \epsilon_n $ must satisfy the boundary condition on the other side of the potential well, namely  that the $ \phi(R)_{0,n} $ remains bounded, which translates into: $ \phi(R)_{0,n} \Rightarrow 0 $ for $ R \Rightarrow \infty $. 

So calculations show (TODO: Verify ) that there are 

\subsubsection{Dipole Transition in 2 Dimension}

The $ D(R) $ is the transition dipole moment  between $ X^2\Sigma^{+} $  and $ A^2\Sigma^{+} $, \begin{equation}
\begin{split}
& D(R) = \Braket{\psi_1(R) | r_1 +  r_2 | \psi_0(R) } \,\,\,\,\,\text{ where in our case : } \\[.8em]
& r_1 + r_2 = \lambda\, R
\end{split}
\end{equation}

Now the 2 dimension case comes into effect. I consider the case where the electrons are confined in 2 dimensions, but where the underlying space is still 3 dimensional, Euclidean. Thus while electrons' movement is confined in 2D, they can radiate photons in any direction in 3D.

TODO: Add appendix for the radiation in 1D and 2D.

Now with the separation of variable, the solution is a product of two function. So the integral above come to:
\begin{equation}\label{dint}
\begin{split}
& D(R) = \int_{-1}^{1}{\int_{1}^{\infty}{ d\mu\, d\lambda\, M_u(\mu)L_u(\lambda)\lambda R M_g(\mu)L_g(\lambda) }}
\end{split}
\end{equation}
where $ M_uL_u $ and $ M_gL_g $ are the odd/even solutions to the original Schrodinger equation \eqref{eqH}.

Since the functions $ M(\mu)L(\lambda) $ and  do not exist in the closed form, the dipole integral \eqref{dint}  is numerically calculated using Wolfram Mathematica.  First the two Mathematica modules (functions) are created, which solve the equations $ L(\lambda) $ and $ M(\mu) $, as function of the internuclear distance $ R $ and corresponding variables. Then the second module calculates the integral above,  as the function of $ R $.

The Mathematica code for the dipole is in appendix TODO: add code and indicate that I wrote it. :)

From the table TODO: add reference one could calculate the spectrum of these transitions.  For the electron to transition from $ A^2\Sigma^{+} $ to $ X^2\Sigma^{+} $ level, it needs to emit photon $ \Delta E \approx  -3.2\,au = 0.1176 eV $, so the corresponding wavelength is $ \lambda \approx   380\,nm $  TODO: Verify the numbers.
TODO: verify the depth of the potential well compared to the paper \cite{H2Plus2d1}

TODO: Possible spectrum of such transitions

\subsubsection{Distorted Wave Approximation}


TODO: Scattering in 2 D

The distorted wave Born approximation (DWBA) is an extension to the (first) Born approximation in scattering processes. Starting from the Schrodinger equation for the scattering problem, we solve it by method of Green's function
\begin{equation}
\begin{split}
& \left[-\frac{\hbar^2}{2m}\nabla^2 + V(\mathbf{r})\right]\psi_k(\mathbf{r}) = E \psi_k(\mathbf{r}) \\[.8em]
& \text{ with } V(\mathbf{r}) = 0 \,\,\,\,\text{ except in the target region }  \Longrightarrow
\end{split}
\end{equation}

The energy $ E $ is the energy of the incident plane wave, $ E = \hbar^2k^2/2m $.  Applying the Green's function

\begin{equation}
\begin{split}
& \left[\frac{\hbar^2}{2m}\nabla^2 + E \right]G_0(\mathbf{r}, \mathbf{r}^{'} | E) = V(\mathbf{r}) \psi_k(\mathbf{r}) 
\end{split}
\end{equation}

we get the integral form of the Schwinger-Lippmann equation:

\begin{equation}\label{SL}
\begin{split}
\psi^{'}_k(\mathbf{r}) = \psi_k(\mathbf{r}) + \int{d^3r^{'}\,G_0(\mathbf{r}, \mathbf{r}^{'}| E)V(\mathbf{r}^{'})\psi^{'}_k(\mathbf{r})}
\end{split}
\end{equation}

The scattering amplitude is given by \cite{DWBA}

\begin{equation}\label{fsc}
f_k(\mathbf{r}) = -\frac{2m}{\hbar^2}\frac{1}{4\pi}\int{d^3r^{'}\,e^{-i\mathbf{k}\mathbf{r}^{'}}V(\mathbf{r}^{'})\psi^{'}_k(\mathbf{r}^{'})}
\end{equation}

Since the Schwinger-Lippmann equation \eqref{SL} is unsolvable, the first Born approximation assumes that the scattered field is small when compared to the incident field. Therefore it treats the scattered way as a perturbation. As the $ 0 - th $ order the scattered wave is an unperturbed incident plane wave: 

\begin{equation}
\begin{split}
\psi^{'}_0(\mathbf{r}) = e^{i\mathbf{k}\mathbf{r}}
\end{split}
\end{equation}

and then the equation \eqref{SL} is solved iteratively:
\begin{equation}\label{SLi}
\begin{split}
\psi^{'}_{n+1}(\mathbf{r}) = e^{i\mathbf{k}\mathbf{r}} + \int{d^3r^{'}\,G_0(\mathbf{r}, \mathbf{r}^{'}| E)V(\mathbf{r}^{'})\psi^{'}_k(\mathbf{r}) }
\end{split}
\end{equation}

So if we expand the wave function in the powers of the interaction potential $ V $ we get:
\begin{equation}\label{SLP}
\begin{split}
& \psi^{'}_k = \psi^{'(0)}_{\mathbf{k}} + \psi^{'(1)}_{\mathbf{k}} +  \psi^{'(2)}_{\mathbf{k}} +\, ...\, \\[.8em]
& = \psi^{'(0)}_{\mathbf{k}} + G_0V\psi^{'(0)}_{\mathbf{k}} + G_0VG_0V\psi^{'(0)}_{\mathbf{k}} + \,...\, \\[.8em]
& = (1 + G_0T)\psi^{'(0)}_{\mathbf{k}}\,\,\,\,\,\,\,\text{ with } T = V + G_0V + .... = \frac{1}{1-VG_0}
\end{split}
\end{equation}

The equation \eqref{SLP} represents a scattering process where incident particle undergoes multiple scattering events from the potential. Since this makes the calculations very complicated, only the first iteration of the series is takes into account and the matrix $ T $ is approximated by the potential $ V $. This first order term in which the exact wave function $ \psi^{'}_{\mathbf{k}}(\mathbf{r}) $ is replaced by the plane wave $ e^{i\mathbf{k}\mathbf{r}} $ is the First Born Approximation. It is very useful, however it is not always valid and one way to extend its validity is the DWBA. 

In the DWBA, we do not assume any more that the scattered field is small compared to the incident field. So in this case, it is possible to generalize the Born approximation. The free space zero potential $ V_0(\mathbf{r}) = 0 $ is replaced by the non-trivial reference potential $ V_1(\mathbf{r}) = 0 $. It is assumed that the scattered wave function $ \psi_{\mathbf{k}}^{'1} $ due to this potential is known, either analytically or numerically as a solution to the following Schwinger-Lippmann equation:

\begin{equation}
\psi^{'}_{n+1}(\mathbf{r}) = e^{i\mathbf{k}\mathbf{r}} + \int{d^3r^{'}\,G_0(\mathbf{r}, \mathbf{r}^{'} | E)V_1(\mathbf{r}^{'})\psi^{'1}_k(\mathbf{r}) }
\end{equation}

Then the interaction potential is treated as a perturbation to the reference potential $ V_1 $, i.e:
\begin{equation}
V(\mathbf{r}) = V_1(\mathbf{r}) + \delta V(\mathbf{R} \,\,\,\,\,\text{ with }\,\,\,\,\left|\delta V\right| << \left|V_1\right|
\end{equation}

So in the DWBA the scattering field is determined by applying the Born approximation:
\begin{equation}
\psi^{'}(\mathbf{r}) = \psi^{'1}(\mathbf{r}) + \int{d^3r^{'}\,G_0(\mathbf{r}, \mathbf{r}^{'}| E)V_1(\mathbf{r}^{'})\psi^{'1}_k(\mathbf{r}) }
\end{equation}
to the scattered wave $ \psi^{'}(\mathbf{r}) $. This distorted wave is the solution of the outgoing-wave Schrodinger equation:

\begin{equation}
\left[\frac{\hbar^2}{2m}\nabla^2 - V_1(\mathbf{r}) + E \right]\psi^{'1}(\mathbf{r}) = 0
\end{equation}
where we can use the Green's function method.

To satisfy the boundary conditions, the asymptotic form of the $ \psi^{'1}(\mathbf{r}) $ for $ r \rightarrow \infty$  is:
\begin{equation}
\psi^{'1}(\mathbf{r}) \rightarrow e^{i\mathbf{k}\mathbf{r}} + \frac{1}{r} e^{ikr}f^1_k(\theta)
\end{equation}
where the scattering amplitude is:
\begin{equation}
f^1_k(\theta) = -\frac{2m}{\hbar^2}\frac{1}{4\pi}\int{d^3r^{'}\,e^{-i\mathbf{k}\mathbf{r}^{'}}V_1(\mathbf{r}^{'})\psi^{'1}_k(\mathbf{r}^{'})}
\end{equation}
This would be the scattering amplitude if the potential $ V_1 $ were the only potential present. The total scattering amplitude is the sum:

\begin{equation}
f_k(\theta) = f^1_k(\theta) + \delta f_k(\theta)
\end{equation}
and $ \delta f_k(\theta) $ is calculated in the Born approximation:

\begin{equation}\label{fDW}
\delta f_k(\theta) \simeq -\frac{2m}{\hbar^2}\frac{1}{4\pi}\int{d^3r^{'}\,\psi^{'1(-)*}_{k'}(\mathbf{r}^{'})V_1(\mathbf{r}^{'})\psi^{'1}_k(\mathbf{r}^{'})}
\end{equation}

The $ \psi^{'1(-)*}_{k'} $ is the known incoming wave, corresponding to the reference potential $ V_1 $ (i.e. solution of the Schrodinger equation) .

The condition for the \eqref{fDW} to be a good approximation is for the $ \delta V(\mathbf{r}) $ to be sufficiently small. What that means is that possible additional scattering does not modify significantly the wave function.

\subsubsection{Laplacian Eigenfunctions and Green's function in 2D}

Starting from the Laplace operator and switching to polar coordinates:

\begin{equation}
\begin{split}
& \nabla^2 = \frac{\partial^2}{\partial\,x^2} + \frac{\partial^2}{\partial\,y^2} = \frac{\partial}{\partial\,r^2} + \frac{1}{r}\frac{\partial}{\partial\,r} + \frac{\partial^2}{\partial\,\theta^2} \\[.8em]
& \nabla^2u(r,\theta) = 0
\end{split}
\end{equation}

If the assume that the solution admits separation of variables:
\begin{equation}
u(r,\theta) = R(r)\Theta(\theta) \Longrightarrow
\end{equation}

the Laplacian transforms into
\begin{equation}
\begin{split}
\frac{R^{''}}{R} + \frac{R^{'}}{rR} + \frac{\Theta^{''}}{r^2\Theta} = -\lambda
\end{split}
\end{equation}

Set $ \frac{\Theta^{''}}{\Theta} = -\gamma $ and we obtain 2 ODEs:
\begin{equation}
\begin{split}
& \Theta^{''} + \gamma^2\Theta = 0 \\[.8em]
& R^{''} + \frac{1}{r}R^{'} + \left(\lambda - \frac{\gamma}{r^2}\right) R = 0
\end{split}
\end{equation}

From this we can see that the solution for the $ \Theta $ equation is:

\begin{equation}
\Theta(\theta) = A\cos\sqrt{\gamma} + B\sin\sqrt{\gamma}
\end{equation}

and using the substitution $ \rho = \sqrt{\lambda}r $ we get the Bessel equation for $ R $.

\begin{equation}
R^{''}_{\rho} + \frac{1}{\rho}R^{'}_{\rho} + \left(1 - \frac{n^2}{\rho^2}\right)R = 0
\end{equation}


